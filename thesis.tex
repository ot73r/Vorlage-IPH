% =========================
% Autor: Michael Mierke
% Erstellt: 08.03.2018
% Aktualisiert: 23.10.2018
% =========================
\documentclass[%
    a4paper,             % Papierformat
    %oneside,            % einseitiger Druck
    %twoside,            % zweiseitiger Druck
    12pt,                % Schriftgröße
    %onecolumn,          % einspaltiger Text
    %twocolumn,          % zweispaltiger Text
    %openright,          % Kapitel dürfen nur auf einer rechten Seite beginnen
    %openany,            % Kapitel dürfen rechts oder links beginnen
    parskip=half,        % eine halbe Zeile Abstand zw. Absätzen
    numbers=noendperiod, % Kapitelnr. nach Unterkapitel deaktivieren
    bibliography=totoc,  % Bibliographie im Inhaltsverzeichnis
    listof=totoc         % Tabellenvz., etc im Inhaltsverzeichnis
    ]{scrreprt}          % KOMA-Script

% ===================================
% Importieren der notwendigen Pakete
% ===================================
\usepackage[automark, headsepline]{scrlayer-scrpage}    % Kopf- und Fußzeilen
\usepackage[utf8]{inputenc}% richtiges Encoding des dokuments (für ß, üöä, etc.)
\usepackage[T1]{fontenc}      % Kodierung in der pdf-Datei
\usepackage[german, english, ngerman]{babel}   % neue deutsche Rechtschreibung
\usepackage{helvet}           % Schriftart: serifenlos
\usepackage{microtype}  % Verbesserung der Silbentrennung und Randausgleich
% Layout der Seite
\usepackage[left=25mm,right=25mm,top=30mm,bottom=20mm]{geometry}
% Zeilenabstände
\usepackage[onehalfspacing]{setspace}
% Grafiken: PDF, GIF, PNG
\usepackage{graphicx}
\usepackage{caption}
\captionsetup{format=hang,font=small,labelfont=bf,
labelsep=colon,justification=justified,singlelinecheck=false}
% Pakete für Blindtexte
\usepackage{lipsum}
\usepackage{blindtext}
\usepackage{xcolor}
% Abkürzungen
\usepackage[printonlyused]{acronym}
% \usepackage{glossaries}
% Hypertextmarks & Metadaten
\usepackage{hyperref}
% Anpassbare Enumerates/Itemizes
\usepackage{enumitem}
\usepackage{csquotes}
% keine widows bzw. clubs
\usepackage[all]{nowidow}
% Pakete für richtige Darstellung der Mathematik
\usepackage{amsmath}
\usepackage{amssymb}

% ========================================================
% Einstellungen für Literaturverzeichnis und Zitationsstil
% ========================================================

\usepackage[%
    backend=biber,
    style=alphabetic,
    language=autobib,
    autolang=other,     % Unterscheidung Volume und Jahrgang
    giveninits=true,
    maxnames=3,
    maxalphanames=1,
    uniquelist=false,
    doi=false
    ]{biblatex}
\addbibresource{thesis.bib}
\defbibheading{lit}{\addchap{Literaturverzeichnis}}

% ============================================
% Anpassungen für Autoren, Title und Publisher
% ============================================

\renewcommand*{\multinamedelim}{\addsemicolon\space} % Trenner zwischen den Namen ein Semikolon
\renewcommand*{\finalnamedelim}{\addsemicolon\space}
\renewcommand *{\labelnamepunct}{\addcolon\space} % Doppelpunkt nach dem letzten Namen
\renewcommand*{\labelalphaothers}{} % entfernt das "+" in der Literaturangabe bei mehreren Autorenthat
\DefineBibliographyStrings{ngerman}{ %ersetzt u.a. with et al. bei mehr als 3 Autoren
     andothers = {{et\,al\adddot}},
  }

% Komma zwischen Publisher und Datum entfernen
\renewbibmacro*{publisher+location+date}{%
  \printlist{location}%
  \iflistundef{publisher}
    {\setunit*{\space}}
    {\setunit*{\addcolon\space}}%
  \printlist{publisher}%
  \setunit*{\space}%
  \usebibmacro{date}%
  \newunit}

% Austausch Vor- und Nachname
\DeclareNameAlias{default}{last-first}

% Anführungszeichen des Titels entfernen
\DeclareFieldFormat
  [book,article,inbook,incollection,inproceedings,patent,thesis,unpublished]
  {title}{#1\isdot}

% ===========================================
% Anpassungen für @article und @inproceedings
% ===========================================

% Unterscheidung Deutsch und Englisch (The short answer is what others have said, % starts a comment that goes to the end of the line. The normal effect is that it doesn't insert the space (or a \par) from the newline.)
\DefineBibliographyExtras{ngerman}{%
  \DeclareFieldFormat[article]{volume}{#1.\addspace Jg.\addspace(\printfield{year}),}%
  \DeclareFieldFormat[inproceedings]{volume}{#1.\addspace Jg.,}%
  \DeclareFieldFormat[article,inproceedings]{issue}{H.\addspace#1}%
}
\DefineBibliographyExtras{english}{%
  \DeclareFieldFormat[article,inproceedings]{volume}{vol.\addspace#1\addspace(\printfield{year}),}%
\DeclareFieldFormat[article]{issue}{no.\addspace#1}%
}

% Journaltitle bzw. Booktitle nicht kursiv
\DeclareFieldFormat[article]{journaltitle}{#1\addcomma}
\DeclareFieldFormat[inproceedings]{booktitle}{#1\addcomma}

% kein Journaltitle anzeigen wenn dieses nicht gegeben wird. (Entfernt das redundante "In:"
% todo

% Publisher wird bei Artikeln standardmäßig nicht angegeben, Lösung über Series
\DeclareFieldFormat[article,inproceedings]{series}{#1\addcomma}

% Entfernt Klammern [parens] bei Datumsangabe und löscht die Jahresangabe
\renewbibmacro*{issue+date}{%
  \printtext[]{%
    \iffieldundef{issue}
      {\printdate}
      {\printfield{issue}%
       \setunit{\addspace}%
       }}%
  \newunit}

% Punkt durch Komma ersetzen hinter der Auflage
\DeclareFieldFormat{edition}{%
    \ifinteger{#1}
      {\mkbibordedition{#1}~\bibstring{edition}}
      {#1\isdot},}

% Macros damit das Komma hinter der Locations verschwindet
\newbibmacro*{pubinstorg+location+date}[1]{%
    \setunit{\addspace}%
    \printlist{#1}%
    \setunit*{\addcomma\space}%
    \printlist{location}%
    \setunit{\addspace}%
    \usebibmacro{date}}%

\renewbibmacro*{publisher+location+date}{\usebibmacro{pubinstorg+location+date}{publisher}}
\renewbibmacro*{institution+location+date}{\usebibmacro{pubinstorg+location+date}{institution}}
\renewbibmacro*{organization+location+date}{\usebibmacro{pubinstorg+location+date}{organization}}

% Anpassungen für Richtigkeit der Onlinequellen
\DeclareFieldFormat{url}{\newline\url{#1}\addcomma}
\DeclareFieldFormat{urldate}{Zuletzt aufgerufen am #1}
\DeclareFieldFormat[online]{title}{#1\addcomma}

% =====================================
% Einstellungen für Kopf- und Fußzeilen
% =====================================
\pagestyle{scrheadings}
\setkomafont{pagehead}{\normalfont\fontsize{10pt}{12pt}}
\setkomafont{pagenumber}{\fontsize{10pt}{12pt}}
\ihead{\leftmark}                               % Kopfzeile innen
\ohead{Seite \pagemark}                         % Kopfzeile außen
\chead{}                                        % Kopfzeile mitte
\cfoot[]{}                                      % Fußzeile leer

\renewcommand*{\chaptermarkformat}{}            % Kopfzeile ohne Kapitelnummer
\renewcommand*{\indexpagestyle}{scrheadings}    % Style der Indexseite
\renewcommand*{\chapterpagestyle}{scrheadings}  % Style der ersten Seite eines Chapters

%------------ Einstellungen der Überschriften ------------
% Schriftgröße der Überschriften aller Ebenen
\setkomafont{chapter}{\fontsize{14pt}{18pt}\selectfont\bfseries}
\setkomafont{section}{\fontsize{13pt}{16pt}\selectfont\bfseries}
\setkomafont{subsection}{\fontsize{12pt}{16pt}\selectfont\bfseries}
\setkomafont{subsubsection}{\fontsize{12pt}{16pt}\selectfont}

% Abstand vor und nach den Überschriften
\RedeclareSectionCommand[%
  beforeskip=24pt,
  afterskip=8pt
  ]{chapter}
\RedeclareSectionCommand[%
  beforeskip=24pt,
  afterskip=8pt
  ]{section}
\RedeclareSectionCommand[%
  beforeskip=24pt,
  afterskip=8pt
  ]{subsection}
\RedeclareSectionCommand[%
  beforeskip=24pt,
  afterskip=8pt
  ]{subsubsection}

% hängender Einzug bei 1,5 cm -> Kapitelname aller Ebenen untereinander
\newlength{\headingindent}
\setlength{\headingindent}{1.5cm}
\renewcommand*{\chapterformat}{%
  \makebox[\headingindent][l]{\thechapter\autodot}%
}
\renewcommand*{\sectionformat}{%
  \makebox[\headingindent][l]{\thesection\autodot}%
}
\renewcommand*{\subsectionformat}{%
  \makebox[\headingindent][l]{\thesubsection\autodot}%
}

% Anpassung der Aufzählungsebenen
\renewcommand{\labelitemi}{---}
\renewcommand{\labelitemii}{---}

\renewcommand{\familydefault}{\sfdefault}   % Schriftart des gesamten Dokuments auf serifenlos

%------------ Einstellungen der Metadaten (Deckblatt etc.) ------------
\newcommand{\pdftitel}{Titel der Arbeit}
\newcommand{\arbeit}{Masterarbeit}
\newcommand{\titel}{\pdftitel}
\newcommand{\autor}{Autor der Arbeit}
\newcommand{\martrikelnr}{Matrikelnummer}
\newcommand{\abschluss}{Master of Science, etc.}
\newcommand{\fakultaet}{Fakultät für Maschinenbau}
\newcommand{\luh}{Leibniz Universität Hannover}
\newcommand{\pruefer}{}
\newcommand{\betreuer}{}

\date{\today}
\begin{document}
% !TEX root = ../thesis.tex

\begin{titlepage}
\begin{spacing}{1}

\sffamily % Schriftart der Titlepage auf serifenlos

% Version 1: Logo und Bezeichnung der Uni bzw. Institut
% \begin{figure}[htbp]
%     \begin{minipage}[t]{50mm}
%         \vspace{0pt}
%         \centering
%         \includegraphics[width=50mm, height=40mm, keepaspectratio]{images/luh_logo.pdf}
%         \end{minipage}
%     \begin{minipage}[t]{30mm}
%         \vspace{0pt}
%         \sffamily   %weil minipage wieder auf serif umstellt
%         \flushleft
%         \fontsize{10pt}{15pt}\selectfont Gottfried Wilhelm Leibniz Universität Hannover
%     \end{minipage}
%     %\hspace{5mm}
%     \begin{minipage}[t]{40mm}
%         \vspace{0pt}
%         \centering
%         \includegraphics[width=\textwidth]{images/IPH_Logo.jpg}
%     \end{minipage}
%     %\hfill
%     \begin{minipage}[t]{35mm}
%         \vspace{0pt}
%         \sffamily
%         \flushleft
%         \fontsize{10pt}{15pt}\selectfont Institut für Integrierte Produktion Hannover gemeinnützige GmbH
%     \end{minipage}
% \end{figure}

% Version 2: Nur Logo
\begin{figure}[htbp]
    \begin{minipage}[b]{60mm}
        \vspace{0pt}
        \centering
        \includegraphics[width=\linewidth]{images/luh_logo.pdf}
        \end{minipage}
    \hfill
    \begin{minipage}[b]{40mm}
        \vspace{0pt}
        \centering
        \includegraphics[width=\linewidth]{images/IPH_Logo.jpg}
    \end{minipage}
\end{figure}


%\enlargethispage{20mm}
\begin{center}
%Abstand davor     \fontsize{Schriftgröße}{Zeilenabstand}
  \vspace*{84pt}	{\fontsize{24pt}{33pt}\selectfont\bfseries\titel }\\
	\vspace*{60pt}	{\fontsize{24pt}{33pt}\selectfont\bfseries\MakeUppercase\arbeit}\\
	\vspace*{60pt}	{\fontsize{14pt}{19pt}\selectfont zur Erlangung des akademischen Grades Master of                          Science an der \fakultaet \,der \luh}\\
	\vspace*{60pt}    {\fontsize{14pt}{19pt}\selectfont vorgelegt von\\}
	\vspace*{48pt}    {\fontsize{12pt}{16pt}\selectfont\autor\\Matr.-Nr.: \martrikelnr}\\
\end{center}
	\vspace{48pt}
	\begin{spacing}{1.2}
	\begin{tabbing}
		mmmmmmmmmmmmmmmmmmmmmmmmmm     \= \kill
		%\textbf{Matrikelnummer, Kurs}  \>  \martrikelnr\\
		\textbf{Erstprüfer:}           \>  \pruefer\\
		\textbf{Betreuer:}              \>  \betreuer\\
		%\vspace*{2mm}
		\vspace{42pt}\\{Hannover, den \today{}}
	\end{tabbing}
	\end{spacing}

\end{spacing}
\end{titlepage}

\newpage\thispagestyle{empty}\quad\newpage % Leere Seite
% Umstellung auf römische Seitenzahlen für toc, lof, lot
\renewcommand{\thepage}{\Roman{page}}
% !TEX root = ../thesis.tex

% Die Zusammenfassung ist in Deutsch und Englisch zu erstellen und beträgt je Umfang
% der Arbeit insgesamt maximal eine Seite. Diese Seite erhält keine Kopfzeile und
% keine Seitennummerierung.

\thispagestyle{empty}
\addtocounter{page}{-3}
\section*{Zusammenfassung}

\lipsum[20]

\section*{Abstract}

\lipsum[20]

\tableofcontents
\listoffigures
\listoftables
% !TEX root = ../thesis.tex
\addchap{Abkürzungsverzeichnis}

\begin{acronym}[slmtA]
    \acro{IoT}{Internet of Things}
    \acro{JDK}{Java Development Kit}
    \acro{PCA}{Principal Component Analysis}
    \acro{KI}{Künstliche Intelligenz}
    \acro{GMM}{Gaussian Mixture Modell}
    \acro{EM}{Expectation-Maximization}
    \acro{ML}{Maximum-Likelihood}
    \acro{CAGD}{Computer Aided Geometric Design}
    \acro{CAD}{Computer Aided Design}
    \acro{CSG}{Constructive Solid Geometry}
    \acro{NURBS}{Non Uniform Rational B-Splines}
    \acro{CNN}{Convolutional Neural Network}
    \acro{SSL}{Semi-Supervised Learning}
\end{acronym}

% \acs{} -> Kurzschreibweise
% \acl{} -> Langschreibweise
% \acp{} -> Plural

% Umstellung auf arabische Seitenzahlen
\renewcommand{\thepage}{\arabic{page}}\setcounter{page}{1}
%!TeX root = ../thesis.tex
\chapter{Einleitung}
\label{sec:einleitung}

\lipsum[20]
Testdokument laut \cite{hage1991}. Und \ac{IoT}.
Englisches Journal: \cite{Muller}a
Deutsche Zeitschrift: \cite{Mullerb}
Buch: \cite{Ciem04}.
\printbibliography[heading=lit]
\newpage
\thispagestyle{empty}
\vspace*{\fill}
\bfseries Eidesstattliche Erklärung:
\normalfont

Hiermit versichere ich, dass ich die vorliegende Arbeit selbstständig verfasst und keine anderen als die angegeben Quellen und Hilfsmittel benutzt habe, dass alle Stellen der Arbeit, die wörtlich oder sinngemäß aus anderen Quellen übernommen wurden, als solche kenntlich gemacht und dass die Arbeit in gleicher oder ähnlicher Form noch keiner Prüfungsbehörde vorgelegt wurde.

\vspace{16mm}
Hannover, \today{}

\end{document}
